\chapter{Introduction}
\label{chap:intro}

% What is single-cell RNA sequencing
Single-cell RNA sequencing is an important tool for analysing the biological processes in the fields such as oncology, developmental biology, neurology, and study of autoimmune and infectious diseases \cite{patel2014single}.
In comparison with its older cousin, bulk RNA sequencing, it provides us with the expression profile at a resolution of individual cells.
In spite of the impossibility of retrieving the full information of the expression profile due to the small number of transcripts in a particular cell, this data can be used to find the patterns of gene expression \cite{lopez2018scvi}.
Using clustering analysis on these high-resolution patterns, we can uncover rare cell types, never observed in the tissue before.
% What are the biases in the single-cell RNA-seq experiment and how do they arise?
However, the interpretation of the single-cell RNA-seq data remains challenging, mainly because of the high dimensionality of such data, limited and variable sensitivity, batch effects and transcriptional noise.
% Modelling of the single-cell RNA-seq data can be used for various tasks such as imputations of missing values, clustering of the different cells and differential expression analysis.
% What was done up to this point?
Several methods have been proposed for this task \cite{pierson2015zifa}.
Unfortunately, most of these methods assume that a generalized linear model can be used to map the single-cell RNA-seq data to a low dimensional representation.
Furthermore, most existing methods cannot be applied to recent huge datasets consisting of many thousands of cells \cite{regev2017science}.

% What will we try to achieve in this work?
In this report, we explore the possibility to model single-cell data by using the variational autoencoder.
This approach allows us to use a non-linear function to map the data to its low dimensional representation.
Furthermore, this representation is probabilistic, which allows us to model the uncertainty of our representation, which might be important information for the downstream analysis.
Also, after the training phase, our model can be used to quickly evaluate new data, which makes it suitable for huge datasets.
% A special type of neural network, which allows us to map single-cell DNA to lower-dimensional latent space.

To introduce variational autoencoder, we need to define and explain a lot of new terminologies.

First, we briefly introduce the concept of neural networks with special attention to autoencoders, because this is the basic machine learning model we are extending in variational autoencoder.

Since our model creates a probabilistic representation, we need to explain the basics of bayesian modelling.
To illustrate the main goal of approximate Bayesian inference, we use the approximate Bayesian computation, because this method is easy to understand.
Other methods aim for the same goal but are using some clever tricks to get there faster.
Next, we compare the most commonly used methods, Markov chain Monte Carlo, to variational inference used in this project and explain the variational inference in details.
Finally, we show how to merge the two concepts of variational inference and autoencoder into variational autoencoder.

% We start by introducing the basics of variational inference and compare it to the classical Bayesian approach.
Afterwards, we present the model created in this project and we show the experiments we performed on well-known MNIST dataset \cite{lecun1998gradient} to test the correctness of our implementation.
Finally, we demonstrate how to use the model on the single-cell RNA-seq data.
